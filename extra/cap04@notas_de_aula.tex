% Filename: cap04@notas_de_aula.tex
% This code is part of 'Notas de aula n\~{a}o oficiais de MS650 e F620'
% 
% Description: This file correspond to part of the textbook using in the course.
% 
% Created: 01.09.12 10:43:23 AM
% Last Change: 01.09.12 10:43:23 AM
% 
% Authors:
% - Raniere Silva, r.gaia.cs@gmail.com
% 
% Copyright (c) 2012, Raniere Silva. All rights reserved.
% 
% This work is licensed under the Creative Commons Attribution-NonCommercial-NoDerivs 3.0 Unported License. To view a copy of this license, visit http://creativecommons.org/licenses/by-nc-nd/3.0/.
%
% This work is distributed in the hope that it will be useful, but WITHOUT ANY WARRANTY; without even the implied warranty of MERCHANTABILITY or FITNESS FOR A PARTICULAR PURPOSE.
%
\chapter{Transformadas de Fourier}
\subsection{Da s\'{e}rie para a Transformada}
Seja a s\'{e}rie de Fourier de uma fun\c{c}\~{a}o com per\'{i}odo $T = 2L$,
\begin{align*}
    f(x) &= \sum_{n = -\infty}^\infty c_n \exp\left( i n \pi x / L \right), \\
    c_n &= \frac{1}{2L} \int_{-L}^L f(x) \exp\left( -i n \pi x / L \right)
    \id{x}.
\end{align*}
O que desejamos agora \'{e} considerar uma fun\c{c}\~{a}o que n\~{a}o seja
necessariamente peri\'{o}dica, ou seja, estender o intervalo para todo o
$\mathbb{R}$ tomando $L \to \infty$. Para isso vamos denotar
\begin{align*}
    K = K(n) &= n \pi / L, \\
    \Delta K &= K(n + 1) - K(n) = \pi / L.
\end{align*}
Ent\~{a}o:
\begin{align*}
    f(x) &= \sum_{n = -\infty}^\infty c_n \exp\left( i K(n) x \right) \\
    &= \sum_{n = -\infty}^\infty \left( \frac{c_n L}{\pi} \right) \exp\left(
    i K(n) x \right) \Delta K
\end{align*}
com
\begin{align*}
    \frac{c_n L}{\pi} &= \frac{1}{2 \pi} \int_{-\infty}^\infty f(x) \exp\left(
    -i K(n) x \right) \id{x}.
\end{align*}

Vamos agora pensar em $n$ como uma fun\c{c}\~{a}o de $K$, ou seja, $n = n(K) = K
L / \pi$. Ent\~{a}o
\begin{align*}
    \frac{c_n L}{\pi} &= c_{K L / \pi} L / \pi = c_L(K)
\end{align*}
e
\begin{align*}
    f(x) &= \sum_{K L / \pi = -\infty}^\infty c_L(K) \exp\left( i K x \right)
    \Delta K, \\
    c_L(K) &= \frac{1}{2 \pi} \int_{-L}^L f(x) \exp\left( - i K x \right)
    \id{x}.
\end{align*}
Agora, uma vez que $\Delta K L / \pi = 1$, para $L \to \infty$ temos $\Delta K
\to 0$ e a soma acima torna-se uma integral de Riemann. Portanto, para $L \to
\infty$:
\begin{align*}
    f(x) &= \int_{-\infty}^\infty c(K) \exp\left( i K x \right) \id{K},
    c(K) &= \frac{1}{2 \pi} \int_{-\infty}^\infty f(x) \exp\left( - i K x
    \right) \id{x}.
\end{align*}
Finalmente, para tornar essas express\~{o}es ``sim\'{e}tricas'', vamos definir
$F(K) = \sqrt{2 \pi} c(-K)$. Assim
\begin{align*}
    f(x) &= \frac{1}{\sqrt{2 \pi}} \int_{-\infty}^\infty F(K) \exp\left( -i K x
    \right) \id{K}, \\
    F(K) &= \frac{1}{\sqrt{2 \pi}} \int_{-\infty}^\infty f(x) \exp\left( i K x
    \right) \id{x},
\end{align*}
onde $F(K)$ \'{e} a transformada de Fourier de $f(x)$, $f(x)$ \'{e} a
transformada de Fourier inversa de $F(K)$, $F(K) = \mathcal{F}\left\{ f(x)
\right\}$ e $f(x) = \mathcal{F}^{-1}\left\{ F(K) \right\}$.
\begin{exem}
    Considere $f(x) = N \exp\left( - \alpha x^2 \right)$ para $\alpha > 0$.
    % TODO Terminar de incluir exemplo da p\'{a}gina 115.
\end{exem}
\begin{exem}
    Considere $f(x) = a / \left( x^2 + a^2 \right)$ para $a > 0$.
    % TODO Terminar de incluir exemplo da p\'{a}gina 116.
\end{exem}
\begin{exem}
    Considere
    \begin{align*}
        f(x) &= \begin{cases}
            1, & |x| \leq a, \\
            0, & |x| > a,
        \end{cases}
    \end{align*}
    para $a > 0$.
    % TODO Terminar de incluir exemplo da p\'{a}gina 117.
\end{exem}
\begin{exem}
    Considere $f(x) = \delta(x)$.
    % TODO Terminar de incluir exemplo da p\'{a}gina 117.
\end{exem}

\section{F\'{o}rmula Integral de Fourier}
Na p\'{a}gina~\pageref{teo:fourier} estudamos o Teorema de Fourier para
s\'{e}ries. Considerando um per\'{i}odo $T = 2L$, esse teorema garante que a
s\'{e}rie
\begin{align*}
    f(x) &= \frac{1}{2L} \int_{-L}^L f(\xi) \id{\xi} + \frac{1}{L} \sum_{n =
    1}^\infty \int_{-L}^L f(\xi) \cos\left( n \pi (\xi - x) / L \right) \id{\xi}
    \\
    &= \lim_{N \to \infty} \int_{-L}^L f(x) \frac{\sin\left( (N + 1 / 2) \pi
    (\xi - x) / L \right)}{2 L \sin\left( \pi (\xi - x) / 2 L \right)}
\end{align*}
quando $f(x)$ \'{e} cont\'{i}nua por partes e com derivadas laterais em $(-L,
L)$ e com per\'{i}odo $2L$.

Vamos agora considerar o limite $L \to \infty$. Para isso vamos supor que exista
a integral
\begin{align*}
    \int_{-\infty}^\infty |f(x)| \id{x} < \infty.
\end{align*}
Nesse caso, temos
\begin{align*}
    \lim_{L \to \infty} \frac{1}{2L} \int_{-\infty}^\infty f(\xi) \id{\xi} = 0
\end{align*}
e agora devemos ver o que acontece com a s\'{e}rie.

Definindo, como na se\c{c}\~{a}o anterior,
\begin{align*}
    K &= N \pi / L, \\
    \Delta K &= \pi / L
\end{align*}
temos
\begin{align*}
    \frac{1}{L} \sum_{n = 1}^\infty \int_{-L}^L f(\xi) \cos\left( n \pi (\xi -
    x) / L \right) \id{\xi} &= \frac{1}{\pi} \sum_{n = 1}^\infty \Delta K
    \int_{-L}^L f(\xi) \cos\left( K(\xi - x) \right) \id{\xi}.
\end{align*}
Para $L \to \infty$ temos $\Delta K \to 0$ e a soma pode ser tomada por uma
integral de Riemann de $K = 0$ at\'{e} $K = \infty$ (pois $K = 0$ para $L \to
\infty$ e $n$ fixo). Logo:
\begin{align*}
    f(x) &= \frac{1}{\pi} \int_0^\infty \int_{-\infty}^\infty f(\xi) \cos\left(
    K(\xi - x) \right) \id{\xi} \id{K}.
\end{align*}
Essa \'{e} a f\'{o}rmula de Fourier. Apesar da natureza heur\'{i}stica as
argumenta\c{c}\~{o}es acima, iremos mostrar agora que ela \'{e} de fato
v\'{a}lida. Para isso precisamremos estender o Lema~\ref{lem:lim_int} da
p\'{a}gina~\pageref{lem:lim_int} de modo a incluir a integral ao longo de toda a
reta, ou seja, precisamos do seguinte:
\begin{lem}
    Seja $F$ uma fun\c{c}\~{a}o cont\'{i}nua por partes e com derivadas lateria
    \`{a} esquerda e \`{a} direita para todo o intervalo real tal que exista a
    integral $\int_{-\infty}^\infty |F(x)| \id{x} < \infty$. Ent\~{a}o
    \begin{align*}
        \lim_{k \to \infty} \int_{-\infty}^\infty F(x) \frac{\sin\left( K(x -
        x_0)
        \right)}{ x - x_0} \id{x} &= \pi \frac{F(x_0 + 0) + F(x_0 - 0)}{2}.
    \end{align*}
\end{lem}
\begin{proof}
    Vamos denotar
    \begin{align*}
        G(x, x_0; K) &= F(x) \frac{\sin\left( k (x - x_0) \right)}{x - x_0} \\
        &= K F(x) \frac{\sin(K (x - x_0)}{K (x - x_0)} \\
        &= K F(x) S\left( K(x - x_0) \right).
    \end{align*}
    Vimos durante a demonstra\c{c}\~{a}o do Lema~\ref{lem:cont}
    (p\'{a}gina~\pageref{lem:cont}) que $|S(x)| \leq 1$ para $x \in \mathbb{R}$;
    logo:
    \begin{align*}
        |G(x, x_0; K) \leq K|F(x)|
    \end{align*}
    e
    \begin{align*}
        \int_{-\infty}^\infty |G(x, x_0; K) \id{x} \leq K \int_{-\infty}^\infty
        |F(x)| \id{x} < \infty
    \end{align*}
    de modo que existe a integral $\int_{-\infty}^\infty G(x, x_0; K) \id{x}.$
    Seja
    \begin{align*}
        H(x_0; K) &= \int_{-\infty}^\infty G(x, x_0; K) \id{x} -
        \frac{\pi}{2} \left[ F(x_0 + 0) + F(x_0 - 0) \right].
    \end{align*}
    % TODO Terminar a demonstra\c{c}\~{a}o a partir da p\'{a}gina 123.
\end{proof}

Agora podemos provar o seguinte:
\begin{teo}
    Seja $f(x)$ cont\'{i}nua por partes e com derivadas laterais \`{a} esquerda
    e \`{a} direita em todo intervalo real tal que $\int_{-\infty}^\infty |f(x)|
    < \infty$. Ent\~{a}o vale a chamada f\'{o}rmula integral de  Fourier:
    \begin{align*}
        \frac{1}{\pi} \int_0^\infty \int_{-\infty}^\infty f(\xi) \cos\left(
        K(\xi - x) \right) \id{\xi} \id{K} &= \frac{1}{2} \left[ f(x + 0) + f(x
        - 0) \right]
    \end{align*}
    para $-\infty < x < \infty$.
\end{teo}
\begin{proof}
    % TODO Terminar a demonstra\c{c}\~{a}o a partir da p\'{a}gina 125.
\end{proof}

Vamos agora escrever a f\'{o}rmula integral de Fourier de uma outra forma.
Usando $\cos(\phi) = \left( \exp(i \phi) + \exp(-i \phi) \right) / 2$,
\begin{align*}
    I &= \frac{1}{2} \left[ f(x + 0) + f(x - 0) \right] \\
    &= \frac{1}{\pi} \lim_{a \to \infty} \int_0^a \int_{-\infty}^\infty f(\xi)
    \left[ \frac{\exp\left( i K(\xi - x) \right) + \exp\left( - i K(\xi - x)
    \right)}{2} \right]
    % Terminar de digitar equa\c{c}\~{o}es.
\end{align*}
o que mostra que na transformada inversa a integral deve ser interpretada no
sentido do valor principal de Cauchy:
\begin{align*}
    \mathcal{F}^{-1}\left[ F(K) \right] &= \frac{1}{\sqrt{2 \pi}} PV
    \int_{-\infty}^\infty F(K) \exp\left( -i K x \right) \id{x} \\
    &= \frac{1}{\sqrt{2 \pi}} \lim_{a \to \infty} \int_{-a}^a F(k) \exp\left(
    -i K x \right) \id{K}.
\end{align*}
\begin{exem}
    Considerando
    \begin{align*}
        f(x) &= \begin{cases}
            0, & x < 0, \\
            \exp\left( -x \right), & x > 0.
        \end{cases}
    \end{align*}
    % TODO Terminar a demonstra\c{c}\~{a}o a partir da p\'{a}gina 127.
\end{exem}

\section{Propriedades das Transformadas de Fourier}
Nota\c{c}\~{a}o: $F(K) = \mathcal{F}\left[ f(x) \right]$.

\subsection{Transla\c{c}\~{a}o}
Temos que
\begin{align*}
    \mathcal{F}\left[ f(x - a) \right] &= \exp\left( i K a \right) F(K), \\
    \mathcal{F}\left[ \exp\left( -i \alpha x \right) f(x) \right] &= F(k -
    \alpha),
\end{align*}
portanto, a transla\c{c}\~{a}o em um espa\c{c}o equivale \`{a}
multiplica\c{c}\~{a}o por uma exponencial (complexa) no espa\c{c}o
rec\'{i}proco.
\begin{proof}
    \begin{align*}
        \mathcal{F}\left[ f(x - a) \right] &= \frac{1}{\sqrt{2 \pi}}
        \int_{-\infty}^\infty f(x - a) \exp\left( i K x \right) \id{x} \\
        &= \frac{1}{\sqrt{2 \pi}} \int_{-\infty}^\infty f(x') \exp\left( i K(x'
        + a) \right) \id{x} \\
        &= \exp\left( i K a \right) F(K), \\ \mathcal{F}\left[ \exp\left( -i
        \alpha x \right) f(x) \right] &= \frac{1}{\sqrt{2 \pi}}
        \int_{-\infty}^\infty f(x) \exp\left( -i \alpha x \right) \exp\left( i K
        x \right) \id{x} \\
        &= \frac{1}{\sqrt{2 \pi}} \int_{-\infty}^\infty f(x) \exp\left( i (k -
        \alpha) x \right) \id{x} \\
        &= F(k - \alpha).
    \end{align*}
\end{proof}

\subsection{Deriva\c{c}\~{a}o}
Temos que
\begin{align*}
    \mathcal{F}\left[ f'(x) \right] &= -i K F(K), \\
    \mathcal{F}\left[ x f(x) \right] &= -i F'(K),
\end{align*}
portanto, a deriva\c{c}\~{a}o em rela\c{c}\~{a}o \`{a} uma vari\'{a}vel equivale
\`{a} ultiplica\c{c}\~{a}o pela outra vari\'{a}vel do espa\c{c}o rec\'{i}proco.
\begin{proof}
    % TODO Terminar a demonstra\c{c}\~{a}o a partir da p\'{a}gina 130.
\end{proof}

A generaliza\c{c}\~{a}o dessas propriedades \'{e} \'{o}bvia:
\begin{align*}
    \mathcal{F}\left[ f^{(n)}(x) \right] &= (-i K)^n F(K), \\
    \mathcal{F}\left[ x^n f(x) \right] &= (-i)^n F^{(n)}(K),
\end{align*}
onde em $\mathcal{F}\left[ f^{(n)}(x) \right]$ devemos upor $\lim_{x \to
\pm\infty} f^{(k)}(x) = 0$ para $k = 0, 1, \ldots, n - 1$.
\begin{obs}
    \'{E} interessante notarmos que as propriedades de transla\c{c}\~{a}o e
    deriva\c{c}\~{a}o se unem atrav\'{e}s do Teorema de Taylor; de fato,
    escrevendo $f(x - a)$ na forma de uma s\'{e}rie de Taylor em torno de $x =
    a$,
    \begin{align*}
        f(x - a) &= \sum_{n = 0}^\infty \frac{f^{(n)}(x)}{n!}(-a)^n
    \end{align*}
    segue, da linearidade de $\mathcal{F}$,
    \begin{align*}
        \mathcal{F}\left[ f(x - a) \right] &= \sum_{n = 0}^\infty
        \frac{(-a)^n}{n!} \mathcal{F}\left [f^{(n)}(x) \right] \\
        &= \sum_{n = 0}^\infty \frac{(-a)^n (-iK)^n}{n!} F(K) \\
        &= \sum_{n = 0}^\infty \frac{(i K a)^n}{n!} F(K) \\
        &= \exp\left( i K a \right) F(K).
    \end{align*}
\end{obs}

\subsection{Identidade de Parseval}
Temos que
\begin{align*}
    \int_{-\infty}^\infty f(x) g^K(x) \id{x} &= \int_{-\infty}^\infty F(K)
    G^K(K) \id{K}.
\end{align*}
\begin{proof}
    % TODO Terminar a demonstra\c{c}\~{a}o a partir da p\'{a}gina 131.
\end{proof}
Uma consequ\^{e}ncia muito importante desse resultado \'{e}:
\begin{align*}
    \int_{-\infty}^\infty |f(x)|^2 \id{x} &= \int_{-\infty}^\infty |F(K)|^2
    \id{K}.
\end{align*}
sendo essa identidade tamb\'{e}m chamada identidade de Parseval. Dessa forma, em
outras palavras:
\begin{align*}
    \| f \|^2 = \| F \|^2.
\end{align*}

\subsection{Teorema da Convolu\c{c}\~{a}o}
% TODO Terminar de incluir arquivo M2S12-10.pdf. Interrompido na p\'{a}gina 132.
% TODO Incluir arquivo M2S12-11.pdf.
% TODO Incluir arquivo M2S12-12.pdf (opcional).
