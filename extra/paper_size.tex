% 'Notas de aula não oficiais de MS650 e F620' (c) 2012, 2013 de Raniere Silva
% <ra092767@ime.unicamp.br>
%
% Este trabalho é baseado nos manuscritos das notas de aula do Professor Doutor
% Jayme Vaz Júnior. para as disciplinas MS650, Métodos de Matemática Aplicada
% II, e F620, Métodos Matemáticos da Física II, disponibilizadas em
% http://www.ime.unicamp.br/~vaz/metodos2S12.htm. É permitido a este fazer uso
% deste trabalho para qualquer fim e sem nenhuma restrição.
%
% É permitido fazer uso das criações do espírito presentes neste trabalho
% diretamente relacionadas com os manuscritos das notas de aula do Professor
% Doutor Jayme Vaz Júnior única e exclusivamente para fins educacionais.
%
% Salvo indicação em contrário, este trabalho foi licenciado com a Creative
% Commons Atribuição-CompartilhaIgual 3.0 Não Adaptada. Para ver uma cópia desta
% licença, visite http://creativecommons.org/licenses/by-sa/3.0/.
%
% Este trabalho encontra-se disponível em
% https://github.com/r-gaia-cs/solucoes_ms650_f620.
%
% Este trabalho é distribuído na esperança que possa ser útil, mas SEM NENHUMA
% GARANTIA; sem uma garantia implícita de ADEQUAÇÃO a qualquer MERCADO ou
% APLICAÇÃO EM PARTICULAR.

% Este arquivo serve para a criação de versões mais adequadas para ereaders e
% tablets.

% Para impressão
\usepackage[top=3cm, bottom=3cm, left=2cm, right=2cm]{geometry}

% Para ereaders (Kindle, Nook, Kobo, ...)
% \usepackage[papersize={160mm,200mm},margin=5mm]{geometry}
% \sloppy

% Para tablets (iPad, GalaxyTab, ...)
% \usepackage[papersize={140mm,190mm},margin=5mm]{geometry}
% \sloppy
