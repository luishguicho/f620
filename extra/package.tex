% Filename: package.tex
% This code is part of 'LaTeX with Vim'.
% 
% Description: This file correspond to the packages to be used.
% 
% Created: 07.06.12 11:30:10 AM
% Last Change: 26.06.12 11:07:21 PM
% 
% Authors:
% - Raniere Silva, r.gaia.cs@gmail.com
% 
% Copyright (c) 2012, Raniere Silva. All rights reserved.
% 
% This work is licensed under the Creative Commons Attribution-ShareAlike 3.0 Unported License. To view a copy of this license, visit http://creativecommons.org/licenses/by-sa/3.0/ or send a letter to Creative Commons, 444 Castro Street, Suite 900, Mountain View, California, 94041, USA.
%
% This work is distributed in the hope that it will be useful, but WITHOUT ANY WARRANTY; without even the implied warranty of MERCHANTABILITY or FITNESS FOR A PARTICULAR PURPOSE.
%
\usepackage[utf8]{inputenc}
\usepackage[T1]{fontenc} 
% \usepackage[top=3cm,left=2cm,right=2cm,bottom=3cm]{geometry}  % Set in the file.
% \usepackage[brazil]{babel}  % Set in the file.
% \usepackage{indentfirst}  % Set in the file.

% Text
\usepackage{enumerate}
\usepackage{latexsym}
\usepackage{parcolumns}
\usepackage[hyphens]{url}
\usepackage{hyperref}
\usepackage{breakurl}
\usepackage[official]{eurosym}

% Tables
\usepackage{multicol}
\usepackage{multirow}
\usepackage{array}

% Math
\usepackage{amsmath}
\usepackage{amsthm}
\usepackage{amsfonts}
\usepackage{amssymb}

\newtheorem{defi}{Defini\c{c}\~{a}o}
\newtheorem{prop}{Proposi\c{c}\~{a}o}
\newtheorem{teo}{Teorema}
\newtheorem{exem}{Exemplo}
\newtheorem{obs}{Observa\c{c}\~{a}o}

\newcommand{\id}[1]{\, \mathrm{d}#1}

% Index
\usepackage{makeidx}
\makeindex

% Figures
\usepackage{pb-diagram}
\usepackage{graphicx, color}
\usepackage{subfig}
\usepackage{tikz}
\usetikzlibrary{patterns}
\usepackage{epsfig}

