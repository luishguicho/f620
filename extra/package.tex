% 'Notas de aula não oficiais de MS650 e F620' (c) 2012, 2013 by Raniere Silva
% <ra092767@ime.unicamp.br>
%
% 'Notas de aula não oficiais de MS650 e F620' is licensed under a
% Creative Commons Attribution-ShareAlike 3.0 Unported License.
%
% You should have received a copy of the license along with this
% work.  If not, see <http://creativecommons.org/licenses/by-sa/3.0/>.

% Lista de pacotes utilizados.

\usepackage[utf8]{inputenc}
\usepackage[T1]{fontenc} 
\usepackage[brazil]{babel}
\usepackage{indentfirst}

% Text
\usepackage{enumerate}
\usepackage{latexsym}
\usepackage{parcolumns}
\usepackage{url}
\usepackage{hyperref}
\usepackage{breakurl}
\usepackage[official]{eurosym}

% Tables
\usepackage{multicol}
\usepackage{multirow}
\usepackage{array}

% Math
\usepackage{amsmath}
\usepackage{amsthm}
\usepackage{amsfonts}
\usepackage{amssymb}
\usepackage{breqn}
\allowdisplaybreaks[4]

\newtheorem{defi}{Definição}
\newtheorem{prop}{Proposição}
\newtheorem{lem}{Lema}
\newtheorem{teo}{Teorema}
\newtheorem{exem}{Exemplo}
\newtheorem{obs}{Observação}

% Deprecated
\newcommand{\id}[1]{\, \mathrm{d}#1}
% Variable of Integration
\newcommand{\vi}[1]{\, \mathrm{d}#1}
% Deprecated
\newcommand{\devp}[2]{\frac{\partial #1}{\partial #2}}
\newcommand{\fder}[2]{\frac{\mathrm{d} #1}{\mathrm{d} #2}}
% Partial DERivative
\newcommand{\pder}[2]{\frac{\partial #1}{\partial #2}}
% Residue (complex analysis)
\DeclareMathOperator{\Res}{Res}

% Index
\usepackage{makeidx}
\makeindex

% Figures
\usepackage{pb-diagram}
\usepackage{graphicx, color}
\usepackage{subfig}
\usepackage{tikz}
\usetikzlibrary{arrows,positioning,fit,petri}
\usetikzlibrary{patterns}
\usepackage{epsfig}
